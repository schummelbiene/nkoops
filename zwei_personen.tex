\section{Zwei-Personen-Nullsummenspiele}
Sei nun $\Gamma = \set{f_{1}, f_{2}, X_{1}, X_{2}}$ mit $f_{2}(x) = -f_{1}(x)$ für alle $x \in X = X_{1} \times X_{2}$. Es werde $f_{1}$ mit $f$ und $f_{2}$ mit $-f$ bezeichnet. Anstelle von $\Gamma$ wie oben schreiben wir $\Gamma = \set{f, X_{1}, X_{2}}$. Ziel von Spieler 1 ist es, $f$ zu minimeren und von Spieler 2, $f$ zu maximieren.   
\begin{definition*}
  Eine Strategiekombination $x^{*} = (x^{*, 1}, x^{*, 2}) \in X $ heißt \markdef{Sattelpunkt} von $f$, falls gilt
  \begin{align*}
    f(x^{*, 1}, x^{2}) \leq f(x^{*, 1}, x^{*, 2}) \leq f(x^{1}, x^{*, 2})
  \end{align*}
für alle $x=(x^{1}, x^{2}) \in X$.  
\end{definition*}
\begin{satz}\label{thm:NGG_saddlepoint}
  Eine Strategiekombination $x^{*} \in X$ ist genau dann ein NGG des Zwei-Personen-Nullsummenspiels $\Gamma = \set{f, X_{1}, X_{2}}$, wenn $x^{*}$ ein Sattelpunkt von $f$ ist. (Beweis: klar.)
\end{satz}
\begin{beispiel} Spieler 1: senkrecht, Spieler 2: waagrecht. 

  \begin{tabular}[h!]{l|c c}
    &1 & 6 \\\hline
    1 & - 3 & 2\\
    2 & - 2 & 1\\
    7 & 3 & -2\\
  \end{tabular}

$X_{1} = \set {1, 2, 7}$, $X_{2} = \set{1, 6}$ $f (x_{1}, x_{2}) = \norm{x_{1} - x_{2}} -3$ soll bezüglich $x_{1}$ minimiert oder bezüglich $x_{2}$ maximiert werden.

Dieses Spiel besitzt kein NGG. Welche Strategien sollen Spieler 1 und 2 wählen?
\begin{itemize}
\item Verlustvermeidungsstrategie für Spieler 1:
  \begin{align*}
    \max_{x_{2}} f(x_{1}, x_{2}) \to \min_{x_{1}} \quad \text{bei } x_{1} \in X_{1}
  \end{align*}
Er sollte also Strategie '2' wählen. Der maximale Verlust von Spieler 1 ist dann
\begin{align*}
  \min_{x_{1} \in X_{1}}\max_{x_{2} \in X_{2}} f(x_{1}, x_{2}) = f(2, 6) = 1
\end{align*}
\item Spieler 2:
  \begin{align*}
    &\min_{x_{1} \in X_{1}} f(x_{1}, x_{2}) \to \max_{x_{2}} \quad \text{bei }x_{2} \in X_{2}. \\
    &\max_{x_{2} \in X_{2}}\min_{x_{1} \in X_{1}} f(x_{1}, x_{2}) = f(7, 6) = -2
  \end{align*}
Spieler 2 verliert dabei (Strategie '6') also maximal 2 ('-2 ist der minimale Gewinn').
\end{itemize}
\end{beispiel}
\begin{voraussetzung}\label{ass:existence} Die folgenden Bedingungen sind erfüllt:
  \begin{enumerate}
  \item Für jedes $x^{1} \in X_{1}$ existiert ein $\hat x^{2} \in X_{2}$, sodass
    \begin{align*}
      f(x^{1}, \hat x^{2}) = \max_{x^{2} \in X^{2}}f(x^{1}, x^{2})
    \end{align*}
existiert. 
\item Für jedes $x^{2} \in X_{2}$ existiert ein $\hat x^{1} \in X_{1}$, sodass
   \begin{align*}
      f(\hat x^{1}, x^{2}) = \min_{x^{1} \in X^{1}}f(x^{1}, x^{2})
    \end{align*}
existiert. 
\item Ferner exisiteren die Ausdrücke $\max_{x^{2} \in X_{2}}\min_{x^{1} \in X^{1}}f(x^{1}, x^{2})$ und $\min_{x^{1} \in X_{1}}\max_{x^{2} \in X^{2}}f(x^{1}, x^{2})$.
  \end{enumerate}
\end{voraussetzung}
\begin{bemerkung*}
  Voraussetzung \ref{ass:existence} sind erfüllt, wenn
  \begin{itemize}
  \item die Strategiemengen $X_{1/2}$ endlich sind oder
  \item $X_{1/2}$ kompakte Teilmengen eines B-Raumes sind und $f$ stetig auf $X_{1} \times X_{2}$ ist.  
  \end{itemize}
\end{bemerkung*}
\begin{lemma}\label{lem:minmax}
  Es sei Voraussetzung \ref{ass:existence} erfüllt. Dann gilt
  \begin{align*}
    \max_{x^{2} \in X_{2}} \min_{x^{1} \in X_{1}} f(x^{1}, x^{2}) \leq \min_{x^{1} \in X_{1}} \max_{x^{2} \in X_{2}} f(x^{1}, x^{2}).
  \end{align*}
\end{lemma}
%\datum{27. April 2015}
\begin{beweis}
  Wegen Voraussetzung \ref{ass:existence} gibt es $\bar x^{2} \in X_{2}$, sodass
  \begin{align*}
    \min_{x^{1} \in X_{1}}f(x^{1}, \bar x^{2}) = \max_{x^{2} \in X_{2}}\min_{x^{1} \in X_{1}}f(x^{1}, x^{2})
  \end{align*}
und es existiert $\bar x^{1} \in X_{1}$, sodass
\begin{align*}
    \max_{x^{2} \in X_{2}}f(\bar x^{1}, x^{2}) = \min_{x^{1} \in X_{1}}\max_{x^{2} \in X_{2}} f(x^{1}, x^{2}).
\end{align*}
Damit folgt
\begin{align*}
  \max_{x^{2} \in X_{2}}\min_{x^{1} \in X_{1}}f(x^{1}, x^{2}) &= \min_{x^{1} \in X_{1}}f(x^{1}, \bar x^{2})\\
&\leq f(\bar x^{1}, \bar x^{2})\\
&\leq  \max_{x^{2} \in X_{2}}f(\bar x^{1}, x^{2})\\
&=\min_{x^{1} \in X_{1}}\max_{x^{2} \in X_{2}} f(x^{1}, x^{2}).
\end{align*}
\end{beweis}
\begin{bemerkung*}
  Die in Lemma \ref{lem:minmax} auftretenden Größen
  \begin{align*}
    &\ubar v \coloneqq \max_{x^{2} \in X_{2}}\min_{x^{1} \in X_{1}} f(x^{1}, x^{2})\\
    &\bar v \coloneqq \min_{x^{1} \in X_{1}}\max_{x^{2} \in X_{2}} f(x^{1}, x^{2})
  \end{align*}
heißen \markdef{unterer Spielwert} und \markdef{oberer Spielwert} des Zwei-Personen-Nullsummenspiels.
\end{bemerkung*}
\begin{satz}
  Es sei Voraussetzung \ref{ass:existence} erfüllt. Dann besitzt das Spiel genau dann ein NGG, wenn der untere und der obere Spielwert übereinstimmen. 
\end{satz}
\begin{beweis}
  Sei $x^{*}$ ein NGG. Dann ist $x^{*}$ auch ein Sattelpunkt von $f$, sodass für alle $x^{1} \in X_{1}$ und $x^{2} \in X_{2}$ 
  \begin{align*}
    f(x^{*, 1}, x^{2}) \leq f(x^{*})\leq f(x^{1}, x^{*, 2})
  \end{align*}
gilt. Dann folgt
\begin{align*}
    \max_{x^{2} \in X_{2}} f(x^{*, 1}, x^{2})  = f(x^{*,1}, x^{*, 2}) = \min_{x^{1} \in X_{1}} f(x^{1}, x^{*,2})
\end{align*}
und weiter
\begin{align*}
  \bar v = \min_{x^{1} \in X_{1}}\max_{x^{2} \in X_{2}} f(x^{1}, x^{2}) \leq  \max_{x^{2} \in X_{2}}  \min_{x^{1} \in X_{1}}f(x^{1}, x^{2}) = \ubar v.
\end{align*}
Zusammen mit Lemma \ref{lem:minmax} erhalten wir $\bar v = \ubar v$.
\vspace{5mm}

Umgekehrt gelte nun $\bar v = \ubar v$, das heißt
\begin{align}\label{eq:minmax}
  \max_{x^{2} \in X_{2}} \min_{x^{1} \in X_{1}}f(x^{1}, x^{2}) = \min_{x^{1} \in X_{1}} \max_{x^{2} \in X_{2}}f(x^{1}, x^{2}).
\end{align}
Sei $x^{*,1} \in X_{1}$ derart, dass 
\begin{align*}
\max_{x^{2} \in X_{2}}f(x^{*,1}, x^{2}) = \min_{x^{1} \in X_{1}} \max_{x^{2} \in X_{2}}f(x^{1}, x^{2})
\end{align*}
 und analog sei $x^{*,2} \in X_{2}$ derart, dass 
 \begin{align*}
   \min_{x^{1} \in X_{1}}f(x^{1}, x^{*,2}) = \min_{x^{1} \in X_{1}} \max_{x^{2} \in X_{2}}f(x^{1}, x^{2}).
 \end{align*}
Wegen \eqref{eq:minmax} gilt dann
\begin{align*}
f(x^{*, 1}, x^{*, 2})\leq \max_{x^{2} \in X_{2}}f(x^{*,1}, x^{2}) = \min_{x^{1} \in X_{1}}f(x^{1}, x^{*,2})\leq f(x^{*, 1}, x^{*, 2}), 
\end{align*}
also Gleichheit. Es folgt 
\begin{align*}
  f(x^{*, 1}, x^{2}) \leq f(x^{*}) \leq f(x^{1}, x^{*, 2})  
\end{align*}
für alle $x^{1} \in X_{1} $ und $x^{2} \in X_{2}$. Daher ist $x^{*}$ ein Sattelpunkt und nach Satz \ref{thm:NGG_saddlepoint} ein NGG.
\end{beweis}
\begin{definition*}
  Stimmen oberer und unterer Spielwert überein, dann heißt $v \coloneqq \bar v = \ubar v$ \markdef{Wert} des Spiels. Das Spiel heißt \markdef{fair}, wenn $v = 0$. 
\end{definition*}
\begin{korollar}
  Es sei Voraussetzung \ref{ass:existence} erfüllt. Ist $x^{*}$ ein NGG, dann gilt $f(x^{*}) = \ubar v = \bar v = v$. 
\end{korollar}

\subsection{Matrixspiele und ihre gemischten Erweiterungen}
\label{sec:matrixspiele}

Für ein Matrixspiel $\Gamma= \set{f, X_{1}, X_{2}}$ seien die Strategiemengen durch $X_{1} = \set{1, \dots, m}$ und $X_{2} = \set{1, \dots, n}$ gegeben. Die Funktion $f$ kann mithilfe einer Matrix $A = (a_{ij}) \in \R^{m \times n}$ vollständig beschrieben werden durch
\begin{align*}
  a_{ij} \coloneqq f(i, j), \qquad i = 1, \dots, m, \quad j = 1, \dots , n.
\end{align*}
Diese Matrix heißt auch \markdef{Auszahlungsmatrix}.
\begin{satz}
  Es gilt:
\enu{\alph}
  \begin{enumerate}
  \item $(i^{*}, j^{*})\in X_{1} \times X_{2}$ ist genau dann ein NGG, wenn
    \begin{align}\label{eq:NGG_matrix}
      a_{i^{*}, j} \leq a_{i^{*}, j^{*}} \leq a_{i, j^{*}}, \qquad \forall i = 1, \dots, m, \quad \forall j = 1, \dots, n
    \end{align}
\item Ein Matrixspiel besitzt genau dann ein NGG, wenn
  \begin{align*}
    \max_{j = 1, \dots, n} \min_{i = 1, \dots, m} a_{ij} = \min_{i = 1, \dots, m} \max_{j = 1, \dots, n} a_{ij} 
  \end{align*}
  \end{enumerate}
\end{satz}
\begin{beweis}
  Folgt direkt aus Satz \ref{thm:NGG_saddlepoint} und Lemma \ref{lem:minmax}. 
\end{beweis}

\begin{definition*}
  Sei $\Gamma = \set{f, X_{1}, X_{2}}$ ein Matrixspiel mit $X_{1} = \set{1, \dots, m}$ und $X_{2} = \set{1, \dots, n}$ und die Auszahlungsfunktion sei $f$, die mithilfe der Auszahlungsmatrix $A$ beschrieben werde. Unter der \markdef{gemischten Erweiterung} des Matrixspiels $\Gamma$ versteht man das Nullsummenspiel $\hat \Gamma = \set{\hat f, \hat X_{1}, \hat X_{2}}$ mit
  \begin{align*}
    \hat X_{1} &\coloneqq \set{x \in \R^{m}_{0, +} |\,\sum_{i = 1}^{m}x_{i} = 1}\\
    \hat X_{2} &\coloneqq \set{y \in \R^{n}_{0, +} |\,\sum_{j = 1}^{n}y_{j} = 1}\\
 \hat f:\, &\hat X_{1} \times \hat X_{2} \to \R \\
&(x, y) \mapsto \sum_{i = 1}^{m} \sum_{j = 1}^{n} x_{i}y_{i}\underbrace{f(i, j)}_{a_{ij}} = x^{T}Ay
  \end{align*}
\end{definition*}
\begin{bemerkung*}
  Die kanonischen Einheitsvektoren in $\hat X_{1}$ und $\hat X_{2}$ werden als \markdef{reine Strategien} bezeichnet. Sie repräsentieren die Strategien des Matrixspiels $\Gamma$. 
\end{bemerkung*}
\begin{itemize}
\item $(x^{*}, y^{*}) \in \hat X_{1} \times \hat X_{2}$ ist genau dann NGG, falls
  \begin{align*}
    (x^{*})^{T} Ay \leq     (x^{*})^{T} Ay^{*} \leq     x^{T} Ay^{*}
  \end{align*}
für alle $x \in \hat X_{1}$, $y \in \hat X_{2}$
\item Ist $(x^{*}, y^{*}) \in \hat X_{1} \times \hat X_{2}$ NGG von $\hat \Gamma$, dann gilt
  \begin{align*}
    (x^{*})^{T} Ay^{*} = \min_{x \in X_{1}} \max_{y \in X_{2}} x^{T}A y =  \max_{y \in X_{2}} \min_{x \in X_{1}}x^{T}A y = 
  \end{align*}
\end{itemize}
%\datum{30. April 2015} fehlt!
%\datum{04. Mai 2015}
\begin{align*}
  T(x, y) &\coloneqq \left( \frac 1 {1 + \sum_{i = 1}^{m}\Phi_{i}(x, y)}\left(x + \sum_{i = 1}^{m}\Phi_{i}(x, y)e^{i}\right), \frac 1 {1 + \sum_{j = 1}^{n}\Psi_{j}(x, y)}\left(y + \sum_{j = 1}^{n}\Psi_{j}(x, y)e^{j}\right)\right)\\
\Phi_{i} (x, y) &= \max {0, x^{T}Ay-(Ay)_{i}}\\
\Psi_{j} (x, y) &= \max {0, x^{T}By-(Bx)_{j}}
\end{align*}
\begin{beweis}
  Sei $(x^{*}, y^{*})$ ein NGG von $\hat \Gamma$. Dann gilt nach Definition:
  \begin{align*}
    &(x^{*})^{T}A y^{*} \leq x^{T}Ay^{*}, \quad \forall x \in \hat X_{1}\\
    &(x^{*})^{T}A y^{*} \leq (x^{*})^{T}By, \quad \forall y \in \hat X_{2}
  \end{align*}
Setzen wir für $x$ bzw. $y$ die Einheitsvektoren des $\R^{m}$ bzw. $\R^{n}$ ein, so folgt
\begin{align*}
&(x^{*})^{T}A y^{*} \leq (Ay^{*})_{i}, \quad \forall i \in \set{1, \dots m}\\
&(x^{*})^{T}A y^{*} \leq (B^{T}x^{*})_{j}, \quad \forall j \in \set{1, \dots, n}
\end{align*}
Dann folgt $\Phi_{1}(x^{*}, y^{*}) = 0$ für $i = 1,\dots, m$ und $\Psi_{j}(x^{*}, y^{*}) = 0$ für $j = 1, \dots, n$, sodass sich dann $T(x^{*},y^{*}) = (x^{*}, y^{*})$ durch Einsetzen ergibt. 

Sei nun $ (x^{*}, y^{*})$ ein Fixpunkt von $T$. Dann gilt insbesondere
\begin{align}\label{eq:star_x}
&  x^{*} = \frac 1 {1 + \sum_{i = 1}^{m}\Phi_{i}(x^{*}, y^{*})}\left(x^{*} + \sum_{i = 1}^{m}\Phi_{i}(x^{*}, y^{*})e^{i}\right)\notag\\
&x^{*} + \sum_{i = 1}^{m}\Phi_{i}(x^{*}, y^{*})x^{*} =x^{*} + \sum_{i = 1}^{m}\Phi_{i}(x^{*}, y^{*})e^{i}\notag \\
&\sum_{i = 1}^{m}\Phi_{i}(x^{*}, y^{*})x^{*} =\sum_{i = 1}^{m}\Phi_{i}(x^{*}, y^{*})e^{i} = (\Phi_{1}(x^{*}, y^{*}), \dots, \Phi_{m}(x^{*}, y^{*}))^{T}, 
\end{align}
Wegen $x^{*} \in \hat X_{1}$ ist $x^{*} \neq 0$. Es sei der Index $I_{0} \in \set{1, \dots, m}$ so gewählt, dass $x_{i_{0}}^{*} \neq 0$ und $(Ay^{*})_{i_{0}}\geq (Ay^{*})_{i}$ für alle $i$ mit $x^{*}_{i} \neq 0$. Dann folgt
\begin{align*}
  (Ay^{*})_{i_{0}} = \sum_{i = 1}^{m}x_{i}^{*}(Ay^{*})_{i_{0}} = \sum_{i = 1, x_{i}^{*}\neq 0}^{m} x_{i}^{*}(Ay^{*})_{i_{0}} \geq \sum_{i = 1}^{m} x_{i}^{*}(Ay^{*})_{i} = (x^{*})^{T}Ay^{*}.
\end{align*}
Daraus ergibt sich $\Phi_{i_{0}}(x^{*}, y^{*}) = 0$. Die $i_{0}$-te Zeile von \eqref{eq:star_x} liefert $\sum_{i = 1}^{m}\Phi_{i}(x^{*}, y^{*})x_{i_{0}} = 0$. Da $x_{i_{0}}^{*} \neq 0$ und $\Phi_{i}(x^{*}, y^{*})\geq 0$ für alle $i = 1, \dots, m$, so folgt $\Phi_{i}(x^{*}, y^{*}) = 0$ für alle $i = 1, \dots, m$. Das impliziert
\begin{align*}
  (Ay^{*})_{i} \geq (x^{*})^{T}Ay^{*}
\end{align*}
für alle $i = 1, \dots, m$ und somit für beliebiges $x \in \hat X_{1}$:
\begin{align*}
  x^{T}Ay^{*} = \sum_{i = 1}^{m} x_{i}(Ay^{*})_{i}\geq\sum_{i = 1}^{m} x_{i}(x^{*})^{T}Ay^{*} = (x^{*})^{T}Ay^{*}.
\end{align*}
Analog kann $(x^{*})^{T}By \geq (x^{*})^{T}By^{*}$ für alle $y \in \hat X_{2}$ gezeigt werden. Also ist $(x^{*}, y^{*})$ ein NGG von $\hat \Gamma$. 
\end{beweis}
\begin{satz}(Fixpunktsatz von Brouwer)

Sei $X \subset \R^{n}$ eine nichtleere, konvexe, kompakte Menge sowie $f: X \to X$ stetig. Dann besitzt $f$ einen Fixpunkt in $X$.
\end{satz}
\begin{korollar}
  Es seien $\Gamma = \set{f_{1}, f_{2}, X_{1}, X_{2}}$ ein Bi-Matrix-Spiel und $\hat\Gamma = \set{\hat f_{1},\hat f_{2},\hat X_{1},\hat X_{2}}$ seine gemischte Erweiterung. Dann besitzt $\hat\Gamma$ mindestens ein NGG.
\end{korollar}
\begin{beweis}
  Sei $X \coloneqq \hat X_{1} \times \hat X_{2}$
  \begin{align*}
T(x, y) &\coloneqq  \frac 1 {1 + \sum_{i = 1}^{m}\Phi_{i}(x, y)}\left(x_{i} + \sum_{i = 1}^{m}\Phi_{i}(x, y)e^{i}\right)\\
&= \frac 1 {1 + \sum_{i = 1}^{m}\Phi_{i}(x, y)}\left(1 + \sum_{i = 1}^{m}\Phi_{i}(x, y)e^{i}\right) = 1
  \end{align*}
Die Abbildung $T$ bildet tatsächlich jedes $(x, y) \in X$ nach $T(x, y) \in X$ ab. $T$ ist stetig und $X$ ist nichtleer, konvex und kompakt. Also liefern Satz 2.10 und Lemma 2.9 die Behauptung.
\begin{figure}[h!]
  \centering
  \begin{tikzpicture}[scale=2]
  \draw[->] (0,0) -- coordinate (x axis mid) (1.2,0);
  \draw[->] (0,0) -- coordinate (y axis mid) (0,1.2);
  
  \draw[dashed](0,1) -- (1,0);
 
  \fill (0,0) circle (1pt) node[below left] {$0$};
  \fill (1,0) circle (1pt) node[below] {$1$};
  \fill (0,1) circle (1pt) node[left] {$1$};

\end{tikzpicture}
  \caption{Veranschaulichung}
  \label{fig:pic}
\end{figure}
\end{beweis}

Wir betrachten nun ein quadratisches Optimierungsproblem:
\begin{align}\label{eq:quadr_prob}
q(x, y, v, w)&  x^{T}Ay + x^{T}By - v-w \to \min_{x, y, v, w}\\
&Ay \geq ve, B^{T}x \geq we, e^{T}x = 1, e^{T}y = 1, x\geq 0, y\geq 0. \notag
\end{align}
\begin{satz}
  Es gelten folgende Aussagen:
  \begin{enumerate}
  \item Für jeden zulässigen Punkt $(x, y, v, w)$ von \eqref{eq:quadr_prob} gilt $q(x, y, v, w)\geq 0$. 
  \item Ist $(x^{*}, y^{*})$ ein NGG von $\hat \Gamma = \set{\hat f_{1}, \hat f_{2}, \hat X_{1}, \hat X_{2}}$, so ist $(x^{*}, y^{*}, v^{*}, w^{*})$ mit $v^{*}\coloneqq (x^{*})^{T}Ay^{*}$ und $w^{*} \coloneqq (x^{*})^{T}By^{*}$ eine Lösung von \eqref{eq:quadr_prob}.
\item Problem \eqref{eq:quadr_prob} ist lösbar und der optimale Zielfunktionswert ist gleich $0$. 
  \end{enumerate}
\end{satz}
\begin{beweis}
  \begin{enumerate}
  \item  Sei $(x, y, v, w)$ zulässig für \eqref{eq:quadr_prob}. Dann gilt
  \begin{align*}
&    x^{T}Ay \geq x^{T}(ve) = vx^{T}e = v, \\
&    x^{T}By \geq (xB^{T})^{T}y = we^{T}y = w, 
  \end{align*}
und damit $q(x, y, v, w)\geq 0$. 
\item Sei $(x^{*}, y^{*})$ ein NGG und seien $v^{*}$ und $w^{*}$ gemäß der Angabe definiert. Dann gilt offenbar $q(x^{*}, y^{*}, v^{*}, w^{*}) = 0$. Nun wird noch gezeigt, dass $(x^{*}, y^{*}, v^{*}, w^{*})$ zulässig für \eqref{eq:quadr_prob} ist. Da $x^{*} \in \hat X_{1}, y^{*} \in \hat X_{2}$ ist bleibt nur noch $Ay^{*} \geq v^{*}e$ und $B^{T}x^{*} \geq w^{*}e$ zu zeigen. Da  $(x^{*}, y^{*})$ ein NGG ist, gilt
  \begin{align*}
v^{*}= (x^{*})^{T}Ay^{*} \leq x^{T}Ay^{*} \quad \forall x \in \hat X_{1}, 
  \end{align*}
speziell für $x = e^{i} \in \R^{m}$ ergibt sich daraus $v^{*}\leq (Ay^{*})_{i}$ und somit $Ay^*\geq v^{*}e$. Analog zeigt man $B^{T}x^{*} \geq w^{*}e$. Also ist $(x^{*}, y^{*}, v^{*}, w^{*})$ zulässig für \eqref{eq:quadr_prob}. Wegen $q (x^{*}, y^{*}, v^{*}, w^{*}) = 0$ und Teil 1. folgt 2.
\item Nach Korollar 2.11 besitzt $\hat\Gamma$ stets ein NGG, welches nach Teil 2 eine Lösung von \eqref{eq:quadr_prob} liefert, wobei der zugehörige Funktionswert von $q$ gleich $0$ ist (vergleiche 2.).
  \end{enumerate}
\end{beweis}
\begin{satz}
Es sei $(x^{*}, y^{*}, v^{*}, w^{*})$ eine Lösung von \eqref{eq:quadr_prob}. Dann gilt: $v^{*} = (x^{*})^{T} Ay^{*}$ und $w^{*} = (x^{*})^{T} By^{*}$ und $(x^{*}, y^{*})$ ist ein NGG von $\hat\Gamma$.
\end{satz}
%\datum{11. Mai 2015}
\begin{beweis}
Beweis von Satz 2.13: Aus Teil (c) des Satzes 2.12 wissen wir, dass $q(x*, y^{*}, v^{*}, w^{*}) = 0$ ist, dass also gilt:  
\begin{align}\label{eq:equality}
  (x*)^{T}Ay^{*} + (x^{*})^{T}By^{*} =  v^{*}+ w^{*}.
\end{align}
Andererseits folgt aus den Nebenbedingungen von \eqref{}, dass
\begin{align*}
  (x^{*})^{T}Ay^{*} \geq v^{*}\cdot e^{T}x^{*} = v^{*}
\end{align*}
und 
\begin{align*}
  (x^{*})^{T}By^{*} = (B^{T}x^{*})^{T} y^{*} \geq w^{*}\cdot e^{T}y^{*} = w^{*}. 
\end{align*}
Zusammen mit \eqref{eq:equality} liefert das $v^{*} = (x^{*})^{T}Ay^{*}$ und $w^{*} = (x^{*})^{T}Ay^{*}$. Seien nun $x \in \hat X_{1}$ und $y \in \hat X_{2}$ beliebig gewählt. Dann folgt
\begin{align*}
  \hat f_{1} (x, y^{*}) = x^{T}Ay^{*} \geq v^{*} = (x^{*})^{T}Ay^{*} = \hat f_{1}(x^{*}, y^{*})
\end{align*}
und
\begin{align*}
  \hat f_{2} (x^{*}, y) = (x^{*})^{T}Ay \geq w^{*} = (x^{*})^{T}Ay^{*} = \hat f_{2}(x^{*}, y^{*}).
\end{align*}
Somit ist $(x^{*}, y^{*})$ NGG von $\hat \Gamma$. 
\end{beweis}
\begin{bemerkung*}
  Zu beachten ist, dass nur eine globale Lösung von \eqref{} ein NGG von $\hat \Gamma$ liefert. Da die Zielfunktion von 2.8 nicht konvex ist, gibt es im Allgemeinen auch lokale Lösungen, deren Zielfunktionswert größer als Null ist. 
\end{bemerkung*}


%%% Local Variables: 
%%% mode: latex
%%% TeX-master: "vorlesung"
%%% End: 
