%\datum{13. April 2015}
\section{Grundlagen}
\subsection{Definition und Beispiele strategischer Spiele}
\begin{definition*}
  Ein \markdef{(strategisches) Spiel (in Normalform)} wird beschrieben durch
  \begin{itemize}
  \item  eine Menge $\set{1, \dots, N}$ von Spielern;
  \item  eine nichtleere Strategiemenge $X_{\nu}$ für jeden Spieler $\nu = 1, \dots, N$;
  \item Funktionen $f_{\nu}: X \to \R$ (sogenannte \markdef{Zielfunktion} des Spielers $\nu$) für $\nu = 1, \dots, N$, wobei $X \coloneqq X_{1} \times \dots \times X_{N}$ die \markdef{Menge aller Strategiekombinationen} bezeichnet. 
  \end{itemize}
Zur Abkürzung für ein solches Spiel schreiben wir
\begin{align*}
  \Gamma \coloneqq \set{f_{\nu}, X_{\nu}}_{\nu = 1}^{N}
\end{align*}
Hierbei ist zunächst noch offen, ob der Spieler $\nu$ seine Zielfunktion minimieren oder maximieren will.
\end{definition*}
\begin{beispiel} Gefangenendilemma

Zwei Gefangene haben gemeinsam eine Straftat verübt. Jeder Einzelne kann die Straftat gestehen oder leugnen.
\begin{itemize}
\item Gestehen beide, so muss jeder fünf Jahre ins Gefängnis.
\item Leugnen beide, kann ihnen die Straftat zwar nicht nachgewiesen werden, aber trotzdem erhält jeder zwei Jahre Gefängnisstrafe wegen unerlaubten Waffenbesitzes. 
\item Gesteht einer von beiden und der andere nicht, so erhält der nicht Geständige zehn Jahre Gefängnisstrafe, während der Geständige aufgrund einer Kronzeugenregelung freigelassen wird.  
\end{itemize}
Modellierung als Spiel:
\begin{itemize}
\item Die beiden Gefangenen sind die Spieler 1 und 2. 
\item $X_{1/2} = \set{L, G}$, wobei $L$ für Leugnen und $G$ für Gestehen steht. 
\item Auszahlungsfunktionen $f_{1/2}: X_{1} \times X_{2} \to \R$. 
\item Der Zielfunktionswert soll minimiert werden (Anzahl der abzusitzenden Jahre). 
\end{itemize}

Vertikal: Spieler 1, horizontal: Spieler 2

\begin{tabular}[h!]{l| c c}
    & L & G \\\hline 
  L & $f_{\nu}(L, L)$ & $f_{\nu}(L, G)$ \\ 
  G &$ f_{\nu}(G, L)$ & $f_{\nu}(G, G)$ 
\end{tabular}

\vspace{3mm}

$\nu = 1$: $f_{1}$
\begin{tabular}[h!]{l| c c}
    & L & G \\\hline 
  L & 2 & 10 \\ 
  G & 0 &  5 
\end{tabular}

$\nu = 2$: $f_{2}$
\begin{tabular}[h!]{l| c c}
    & L & G \\\hline 
  L &  2& 0 \\
  G & 10& 5
\end{tabular}

\vspace{3mm}

Also gesamt: 
\begin{tabular}[h!]{l| c c}
    & L & G \\\hline 
  L &  (2, 2)& (10, 0) \\
  G & (0, 10)&  (5, 5)
\end{tabular}

 mit z.B. $(10, 0) = (f_{1}(L, G), f_{2}(L, G))$.

Ändert nur ein Spieler seine Meinung nach zuvor 'fairem' Abkommen, so verschlechter sich seine Lage, die des anderen wird dagegen besser.
\end{beispiel}
\begin{beispiel} Kampf der Geschlechter

Ein Paar plant einen gemeisamen Abend. Die Auswahl besteht zwischen Fußballspiel im Stadion, Theater (oder ihn doch getrennt verbringen). 

Modellierung als Zwei-Personen-Spiel:
\begin{itemize}
\item Spieler 1 und 2 sind Mann und Frau. 
\item $X_{1/2} = {F, T}$ mit $F$ für Fußball und $T$ für Theater. 
\item Auszahlungsfunktionen $f_{\nu}: X_{1} \times X_{2} \to \R$

Mann: vertikal, Frau vertikal

  \begin{tabular}[h!]{l| c c}
    & F & T \\\hline 
    F & (2, 1)& (0, 0) \\
    T & (0, 0)&  (1, 2)
  \end{tabular}
\item Das Ziel der Spieler ist es, ihre Auszahlung zu maximieren. 
\item Eine Umentscheidung eines Spielers nach einem guten Einigung verschlechtert das Ergebnis für beide Spieler.
\end{itemize}
\end{beispiel}
\begin{beispiel} Stein-Schere-Papier
  Zwei Spieler haben zwischen Stein, Schere und Papier zu wählen. Papier schlägt Stein, Stein schlägt Schere und Schere schlägt Papier. Der Gewinner erhält einen Euro. Modellierung als Zwei-Personen-Spiel:
  \begin{itemize}
  \item $X_{1/2} = \set{R,P,S}$, die Auswahl besteht zwischen $R$ (rock), $P$ (paper) oder $S$ (scissor). 
    \item Auszahlungsfunktionen;
vertikal: Spieler 1, horizontal: Spieler 2

  \begin{tabular}[h!]{l| c c c}
    & R & S & P \\\hline 
    R & (0, 0)& (1, -1) & (-1, 1) \\
    S & (-1, 1)&  (0, 0)&  (1, -1)\\ 
    P & (1, -1)&  (-1, 1)&  (0, 0)
  \end{tabular}
\item Jeder Spieler will seine Auszahlung maximieren.
  \end{itemize}
\end{beispiel}

\begin{beispiel} Oligopol-Modell nach Cournot
  
Ein Produkt werde von $N$ Unternehmen hergestellt. Es wird angenommen, dass es keine weiteren Hersteller dieses Produktes gibt. Es werde mit $x_{\nu}$ die vom Unternehmen $\nu$ hergestellte Menge bezeichnet. Der auf dem Markt zu erziehlende Preis für eine Einheit des Produkts sei gegeben durch
\begin{align*}
  p(x_{1}, \dots, x_{N}) = b-(x_{1} + \dots + x_{N})
\end{align*}
mit einer Konstanten $b > 0$. Mit $K_{\nu}(x_{\nu})$ werden die Kosten zur Herstellung von $x_{\nu}$ Einheiten des Produkts durch Hersteller $\nu$ bezeichnet. Beispielsweise kann das sein:
\begin{align*}
  K_{\nu}(x_{\nu}) \coloneqq k \, x_{\nu} \\
  K_{\nu}(x_{\nu}) \coloneqq k \, x_{\nu}^{2}
\end{align*}
mit festem $k > 0$. 

Das Unternehmen möchte seinen Gewinn an Herstellung und Verkauf des Produktes maximieren. 

Modellierung als $N$-Personen-Spiel: 
\begin{itemize}
\item Spieler $1, \dots, N$ sind die $N$ Unternehmen. 
\item $x_{1} = x_{2} = \dots = x_{N} = [0, \infty)$ (eindimensionale Strategie: kleines $x$)
\item Auszahlungsfunktionen:
  \begin{align*}
    f_{\nu}(x_{1}, \dots, x_{N}) \coloneqq x_{\nu}\cdot p(x_{1}, \dots, x_{N}) - K_{\nu}(x_{\nu})
  \end{align*}
\item Jeder Spieler will seinen Gewinn $f_{\nu}$ maximieren. 
\end{itemize}
\end{beispiel}
\subsection{Nash-Gleichgewichte}
Es sei $\Gamma = \set{f_{\nu}, X_{\nu}}_{\nu = 1}^{N}$ und $f_{\nu}$ soll minimiert werden für alle $\nu$. Ein Element aus $X_{\nu}$ werde mit $x^{\nu}$ bezeichnet (oder mit $x_{\nu}$, falls $X_{\nu} \subseteq \R$). Ferner sei
\begin{align*}
  x = (x_{1}, \dots, x_{N}) \in X_{1} \times \dots \times X_{N}.
\end{align*}
\begin{definition*}
  Eine Strategiekombination $x^{*} = (x^{*, 1}, \dots, x^{*, N}) \in X$ heißt \markdef{Nash-Gleichgewicht} des Spiels $\Gamma$, wenn gilt, dass
  \begin{align}\label{eq:nash}
    f_{\nu}(x^{*}) \leq f_{\nu}(x^{*, 1}, \dots, x^{*, \nu-1}, x^{\nu}, x^{*, \nu+1}, \dots, x^{*, N}) \qquad \forall x^{\nu} \in X_{\nu}.
  \end{align}
Das Problem, ein Nash-Gleichgewicht (NGG) eines Spieles zu finden, wird als \markdef{Nash-Gleichgewichtsproblem} (Nash equilibrium problem, NEP) bezeichnet. Ein NGG $x^{*}$ ist also eine Strategiekombination, bei der sich keiner der Spieler verbessern kann, indem er \emph{als Einziger} von der Strategie $x^{*, \nu}$ abweicht. 
\end{definition*}
%\datum{16. April 2015}
Bemerkungen:
\begin{enumerate}
\item Schreibweise $x = (x^{1}, \dots, x^{N}) = (x^{1}, \dots, x^{\nu-1}, x^{\nu}, \dots, x^{N})= (x^\nu, x^{-\nu})$, $x^{-\nu} =(x^{1}, \dots, x^{\nu-1}, x^{\nu+1}, \dots, x^{N}) $. Damit lässt sich \eqref{eq:nash} schreiben als
  \begin{align*}
    f_{\nu}(x^{*})\leq f_{\nu}(x^{\nu}, x^{*, -\nu}) 
  \end{align*}
für alle $x^{0} \in X_{\nu}$ oder 
\begin{align*}
      f_{\nu}(x^{*, \nu}, x^{*, -\nu})\leq f_{\nu}(x^{\nu}, x^{*, -\nu}) \forall x^{\nu} \in X_{\nu} 
\end{align*}
\item $x^{*} \in X$ ist offenbar genau dann ein NGG, wenn $x^{*, \nu}$ für $\nu = 1, \dots, N$ Lösung der folgenden OA ist:
  \begin{align}\label{eq:ngg_equ}
    f_{\nu}(x^{\nu}, x^{*, -\nu}) \to \min_{x^{\nu}} \qquad x^{\nu} \in X_{\nu}
  \end{align}

\end{enumerate}
\begin{definition}
  Für jedes $\nu = 1, \dots, N$ und jedes $x^{-\nu} \in X_{-\nu}\coloneqq \prod_{\eta = 1, \eta \neq \nu}^{N} X_{\eta}$ definieren wir durch
  \begin{align*}
    S_{\nu}(x^{-\nu}) \coloneqq \set{x^{\nu}\in X_{\nu}\setminus f_{\nu}(x^{\nu}, x^{-\nu}) \leq f_{\nu}(y^{\nu}, x^{-\nu}) \text{für alle } y^{\nu} \in X_{\nu}} 
  \end{align*}
die Menge der besten Antworten des Spielers $\nu$ auf die Strategiekombination $x^{-\nu}$ der Gegenspieler. Die mengenwertige Abbildung $x^{-\nu}\mapsto S_{\nu}(x^{-\nu})$ heißt \markdef{Beste-Antwort-Funktion}.
\end{definition}
\begin{satz}
  Eine Strategiekombination $x^{*} \in X$ ist genau dann ein NGG des Spiels, wenn $x^{*}$ Fixpunkt der mengenwertigen Abbildung $x \mapsto S(x)$ ist, das heißt, wenn $x^{*} \in S(x^{*})$ gilt. 
\end{satz}
  \begin{beweis}
Sei $x^{*} \in X$ beliebig gewählt. Unter Beachtung der Definition des NGG sowie der Mengen $S_{\nu}(x^{*, - \nu})$ und $S(x^{*})$ ergibt sich 

$x^{*}$ ist NGG $\iff$
\begin{align*}
  \forall \nu \in \set{1, \dots, N} \forall x^{\nu}\in X_{\nu}: \, &f_{\nu}(x^{*, \nu}, x^{*, -\nu})\leq f_{\nu}(x^{\nu}, x^{*, -\nu}) \\
\iff &\forall \nu \in \set{1, \dots, N}: \, x^{*, \nu}\in S_{\nu}(x^{*, -\nu}) \\
\iff &x^{*} \in S(x^{*}).
\end{align*}
\end{beweis}

\begin{beispiel*}Fortsetzung

  \begin{itemize}
  \item Gefangenendilemma: NGG: $x^{*} = (G, G)^{T}$
  \item Fußball vs. Theater: NGG $x^{*} \in \set{(F, F)^{T}, (T, T)^{T}}$
  \item Stein, Schere, Papier: es gibt kein NGG
  \item Sei $N = 2$ (Duopol), $K_{\nu}(x_{\nu}) = k x_{\nu}^{2}$ mit $k > 0 $ fest, $p(x_{1}, x_{2} = b - (x_{1} + x_{2}))$ mit $b > 0$ fest.
    \begin{align*}
      f_{\nu}(x_{1}, x_{2}) = x_{\nu}(b - x_{1}- x_{2}) - k x_{\nu}^{2}
    \end{align*}
für $\nu = 1,2$. Sei zunächst $x_{2} \in X_{2} \in [0, \infty)$ gegeben. Wir ermitteln nun $S_{1}(x_{2})$. Dazu ist $f_{1}(x_{1}, x_{2}) \max_{x_{1}}$ zu lösen bei $x_{1} \in [0, \infty)$.
\begin{align}\label{eq:h_opt}
  h_{1}(x_{1}, x_{2}) = - (k + 1)x_{1}^{2} + x_{1}(b - x_{2}) \to \max_{ x_{1}} \in [0, \infty). 
\end{align}
Falls $x_{2} \geq b$, so ist $S_{1}(x_{2}) = \set 0$. Falls $x_{2} < b$ dann ist $\frac {\pd h(x_{1}, x_{2})}{\pd x_{1}} = 0$ notwendig und hinreichend für Optimalität, das heißt
\begin{align*}
&  - 2 x_{1}(k + 1) + b - x_{2} = 0\\
&  \frac{b - x_{2}}{2(k + 1)} = x_{1}
\end{align*}
also $S_{1}(x_{2}) = \set{ \frac{b - x_{2}}{2(k + 1)} = x_{1}}$ und zusammenfassend
\begin{align*}
&  S_{1}(x_{2}) =
  \begin{cases}
    \set 0 & x_{2}\geq b \\
     \set{\frac{b - x_{2}}{2(k + 1)}} & x_{2}< b
  \end{cases}\\
  S_{2}(x_{1}) =
  \begin{cases}
    \set 0 & x_{1}\geq b \\
     \set{\frac{b - x_{1}}{2(k + 1)}} & x_{1}< b
  \end{cases}.
\end{align*}
Nach Satz 1.1 muss für ein NGG $x^{*}$ gelten, dass $x^{*} \in S(x^{*})$, das heißt $x_{1}^{*} \in S_{1}(x_{2}^{*})$, $x_{2}^{*} \in S_{2}(x_{1}^{*})$. Angenommen, es sei $x_{1}^{*}\geq b$. Dann folgt $S_{2}(x_{1}^{*}) = \set 0$, $S_{1}(0) = \set{\frac{b}{2(k + 1)}}$, $\frac b {2(k + 1)}< b$. Das ist ein Widerspruch zu $x_{1}^{*}\geq b$ ($x_{2}^{*} \geq b$ analog).
Sei nun $x_{1}<b$ und $x_{2}<b$. Man versucht nun eine Lösung $(x_{1}^{*}, x_{2}^{*})$ des folgenden Systems zu finden:
\begin{align*}
  x_{1} = \frac{b-x_{2}}{2(k + 1)}\\
  x_{2} = \frac{b-x_{1}}{2(k + 1)}.
\end{align*}
Es gibt genau eine Lösung $(x_{1}^{*}, x_{2}^{*}) = \left(\frac{b}{2k + 3}, \frac{b}{2k + 3} \right)$
  \end{itemize}
\end{beispiel*}
\subsection{Dominierte und dominante Strategien}
\begin{definition}
  Eine Strategie $x^{\nu} \in X_{0}$ \markdef{dominiert} eine Strategie $y^{\nu} \in X_{\nu}$ des Spielers $\nu$, wenn für alle $x^{-\nu} \in X_{-\nu}$ die Ungleichung
  \begin{align*}
    f_{\nu}(x^{\nu}, x^{-\nu})\leq f_{\nu}(y^{\nu}, x^{-\nu})
  \end{align*}
erfüllt ist. Äquivalent sagt man, die Strategie $y^{\nu}$ wird von der Strategie $x^{\nu}$ dominiert. Eine Strategie $x^{\nu} \in X_{\nu}$ heißt dominant für den Spieler $\nu$, wenn sie jede andere Strategie dieses Spielers dominiert. Eine Strategiekombination $x^{*} \in X$ heißt \markdef{Gleichgewicht in dominanten Strategien}, wenn für jeden Spieler $\nu = 1, \dots, N$ die Strategie $x^{*, \nu}$ dominant ist, wenn also für alle $\nu = 1, \dots, N$ gilt:
\begin{align*}
  f_{\nu}(x^{*, \nu}, x^{-\nu}) \leq f_{\nu}(x^{\nu}, x^{-\nu}) \quad \forall x = (x^{\nu}, x^{-\nu}) \in X. 
\end{align*}
\end{definition}
\begin{satz}
  Eine Strategie $\bar x^{\nu} \in X_{\nu}$ ist dominant für den Spieler $\nu$ genau dann, wenn $\bar x^{\nu} \in S_{\nu}(x^{- \nu})$ für alle $x^{-\nu} \in X_{- \nu}$, wenn also $x^{-\nu}$ die beste Antwort auf jede mögliche Strategiekombination der Gegenspieler ist. Eine Strategiekombination $x^{*} \in X$ ist genau dann ein GG in dominanten Strategien, wenn $x^{*} \in S(x)$ für alle $x \in X$.  
\end{satz}
%\datum{20. April 2015}
\begin{beispiel}
Zwei Spieler drücken Knöpfe.
vertikal: Spieler 1, horizontal: Spieler 2

  \begin{tabular}[h!]{l| c c c}
    & blau & rot  \\\hline 
    blau & (2, 1)& (9, 0) \\
    rot & (1, 0)&  (9, 8)
  \end{tabular}  

Ziel: maximale Auszahlung.
\begin{itemize}
\item Die blaue Strategie dominiert für Spieler 1 die rote, denn der Auszahlungsbetrag der roten ist kleiner oder gleich der blauen. Also streicht er die rote Strategie. 
\item Spieler 2 besitzt ursprünglich keine dominierte Strategie. Nach Streichung der Strategie 'rot' für Spieler 1 dominiert die blaue Strategie die rote bei Spieler 2. Folglich kann er die Strategie 'rot' streichen. \item Es ergibt sich die Kombination (blau, blau). Diese ist ein Nash-Gleichgewicht, das ist kein Zufall. 
\end{itemize}
\end{beispiel}
\begin{satz}
\renewcommand{\labelenumi}{(\alph{enumi})}
  \begin{enumerate}
  \item Ein Gleichgewicht in dominanten Strategien ist auch ein Nash-Gleichgewicht. 
\item Falls nach einer endlichen Anzahl von Eliminationen von dominierten Strategien genau eine Strategiekombination übrig bleibt, so ist diese ein Nash-Gleichgewicht. 
  \end{enumerate}
\end{satz}
\begin{beweis}
  Übung.
\end{beweis}
\begin{bemerkung*}
  Im Allgemeinen ist ein NGG kein GG in dominanten Strategien. Es kann NGGs auch in dominierten Strategien geben (oben (rot, rot)). 
\end{bemerkung*}
\subsection{Klassifikation von Spielen}
\begin{definition*}
  Ein Spiel $\Gamma=\set{f_{\nu}, X_{\nu}}_{\nu = 1}^{N}$ heißt
  \begin{itemize}
  \item \markdef{endlich}, wenn alle $X_{\nu}$ höchstens endlich viele Elemente besitzen, 
  \item \markdef{abzählbar}, wenn alle $X_{\nu}$ höchstens abzählbar viele Elemente enthalten. 
  \end{itemize}
\item \markdef{überabzählbar} oder \markdef{kontinuierlich}, wenn mindestens eine Strategiemenge überabzählbar viele Elemente besitzt. 
\end{definition*}
\begin{definition*}
  Ein Spiel $\Gamma$ heißt \markdef{Nullsummenspiel}, wenn für alle $x \in X$
  \begin{align*}
    \sum_{\nu = 1}^{N} f_{\nu}(x) = 0.
  \end{align*}
Sonst heißt es \markdef{Plus-/Minus-Summenspiel}. 
\end{definition*}
 Endliche \markdef{Zwei-Personen-Spiele} heißen auch \markdef{Bi-Matrix-Spiele}. Seien $x_{1}, \dots, x_{m}$ die Strategien von Spieler 1 und $y_{1}, \dots, y_{n}$ die Strategien von Spieler 2. Dann lassen sich die Auszahlungsfunktionen $f_{1/2}$ vollständig durch die Matrizen
 \begin{align*}
&   A \coloneqq \left(f_{1}(x_{i},y_{j}) \right)_{ij} \in \R^{m \times n} \quad \text{und}\\
&   B \coloneqq \left(f_{2}(x_{i},y_{j}) \right)_{ij} \in \R^{m \times n}
 \end{align*}
beschrieben werden. In endlichen Zwei-Personen-Nullsummenspielen gilt $B = -A$, sodass dann die Matrix $A$ zur Beschreibung des Spiels genügt. Endliche Zwei-Personen-Spiele heißen auch \markdef{Matrix-Spiele}.
\begin{itemize}
\item Kooperative Spiele vs. \markdef{nicht-kooperative} Spiele (Absprachen in kooperativen Spielen sind bindend)
\item \emph{statische} vs. dynamische Spiele (dynamische sind zeitabhängig)
\item Spiele mit \emph{vollständiger} vs. unvollständiger Information
\end{itemize}

%%% Local Variables: 
%%% mode: latex
%%% TeX-master: "vorlesung"
%%% End: 
